\documentclass{article}
\usepackage{listings}
\usepackage{xcolor}


\definecolor{codegreen}{rgb}{0,0.6,0}
\definecolor{codegray}{rgb}{0.5,0.5,0.5}
\definecolor{codepurple}{rgb}{0.58,0,0.82}
\definecolor{backcolour}{rgb}{0.95,0.95,0.92}

\lstdefinestyle{mystyle}{
    backgroundcolor=\color{backcolour},   
    commentstyle=\color{codegreen},
    keywordstyle=\color{magenta},
    numberstyle=\tiny\color{codegray},
    stringstyle=\color{codepurple},
    basicstyle=\ttfamily\footnotesize,
    breakatwhitespace=false,         
    breaklines=true,                 
    captionpos=b,                    
    keepspaces=true,                 
    numbers=left,                    
    numbersep=5pt,                  
    showspaces=false,                
    showstringspaces=false,
    showtabs=false,                  
    tabsize=2
}

\lstset{style=mystyle}

\begin{document}

\section{Question 1}


https://www.w3schools.com/python/ref_string_find.asp and i uses chatgpt as well. 



\section{Question 2}


\subsection{What is a data type?}

A data type defines the kind of data a variable can hold and the operations that can be performed on it. 

It could be an string like "Hallo" or an interger "3" or a float "3.2" 

Data types can help with structuring data, make working with data more efficient. For example with optimize memory usage so there is allocated the correct amount of space. It can also help with minimaze "Error Prevention" so that two different types can't combine.

an example is from the exercise week 13, question 1. "How do you choose to represent the price of electricity? Describe three possible data structures (using Python) that you could use and explain which one you choose."


the way i would have solved it was





\subsection{Representing colours}

There it multiple ways (RGB: Best for precise control of color components, Hexadecimal: Convenient for web design and compatibility with CSS) 


examples of this:

color_rgb = (255, 0, 0)  # Red color
color_hex = "#FF0000"  # Red color
color_name = "red"  # Red color


\subsection{Representing Neighbour Relation}




\subsection{Colouring problem}


\section{ Question 3}


\subsection{}



\lstinputlisting[language=Octave]{scr/diku.py}

\subsection{}



Immutability:

Functional programming relies heavily on immutable data. In the case of the function, the text and word are not altered within the function, which can reduce bugs related to unintended side effects from state changes.

Debugging and Testing:

Pure functions are easier to test and debug because their output depends solely on the input. There’s no hidden state that can change unexpectedly, which helps in understanding and validating the program's behavior.


\subsection{}





\section{Question 4}
\subsection{}


Here I define two functions, one called \texttt{calculate\_points} and another called \texttt{total\_points\_from\_file}.

The first function, \texttt{calculate\_points}, takes as input a string.

The second function, \texttt{total\_points\_from\_file}, takes as input the name of a \texttt{.txt} file. First, it initializes a variable \texttt{total} to zero. Then, it uses the built-in \texttt{open()} function on the filename provided as input to the function. It splits each line into a maximum of two parts using the colon (\texttt{:}) as the delimiter, and takes the second element of the list, removes any leading and trailing whitespace using \texttt{.strip()}, and saves this string as \texttt{card\_data}. After this, it appends the value from \texttt{calculate\_points(card\_data)} to \texttt{total}, adding the points from one string. The loop continues until there are no more lines to process, thanks to the \texttt{for line} statement.

\lstinputlisting[language=Octave]{scr/cards.py}

\subsection{}

What programming paradigm dominates in your program. Why?

here i use object ortineted programming, also since the build in methoes is used in this way.


\subsection{}


Explain how you test your program.


\section{}




\subsection{}

\end{document}