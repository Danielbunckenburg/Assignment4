\documentclass{article}
\usepackage{listings}
\usepackage{xcolor}


\definecolor{codegreen}{rgb}{0,0.6,0}
\definecolor{codegray}{rgb}{0.5,0.5,0.5}
\definecolor{codepurple}{rgb}{0.58,0,0.82}
\definecolor{backcolour}{rgb}{0.95,0.95,0.92}

\lstdefinestyle{mystyle}{
    backgroundcolor=\color{backcolour},   
    commentstyle=\color{codegreen},
    keywordstyle=\color{magenta},
    numberstyle=\tiny\color{codegray},
    stringstyle=\color{codepurple},
    basicstyle=\ttfamily\footnotesize,
    breakatwhitespace=false,         
    breaklines=true,                 
    captionpos=b,                    
    keepspaces=true,                 
    numbers=left,                    
    numbersep=5pt,                  
    showspaces=false,                
    showstringspaces=false,
    showtabs=false,                  
    tabsize=2
}

\lstset{style=mystyle}

\begin{document}

\section{Question 1}


\section{Question 2}


\subsection{}
A data type is 

Illustrate the role data types play when representing data in Python, using examples from the thursday worksheets in 


\subsection{}

You can repersent couler as number from 1 to n, numbers each number repersent a couler. The problem with this stragerty is that you might run out of destitied colur so it can be hard to tell the couler apart.  


you can also do it by only giving 4 different coulre. 



\newpage

\section{}






The next code will be directly imported from a file as an example of an exercise


\lstinputlisting[language=Octave]{scr/main_program.py}





\section{}

\subsection{}


\subsection{}

What programming paradigm dominates in your program. Why?





I use function programming, 


\subsection{}

Explain how you test your program.


\section{}

\subsection{}

\end{document}