\documentclass{article}
\usepackage{listings}
\usepackage{xcolor}


\definecolor{codegreen}{rgb}{0,0.6,0}
\definecolor{codegray}{rgb}{0.5,0.5,0.5}
\definecolor{codepurple}{rgb}{0.58,0,0.82}
\definecolor{backcolour}{rgb}{0.95,0.95,0.92}

\lstdefinestyle{mystyle}{
    backgroundcolor=\color{backcolour},   
    commentstyle=\color{codegreen},
    keywordstyle=\color{magenta},
    numberstyle=\tiny\color{codegray},
    stringstyle=\color{codepurple},
    basicstyle=\ttfamily\footnotesize,
    breakatwhitespace=false,         
    breaklines=true,                 
    captionpos=b,                    
    keepspaces=true,                 
    numbers=left,                    
    numbersep=5pt,                  
    showspaces=false,                
    showstringspaces=false,
    showtabs=false,                  
    tabsize=2
}

\lstset{style=mystyle}

\begin{document}

\section{Question 1}


https://www.w3schools.com/python/ref_string_find.asp and i uses chatgpt as well. 



\section{Question 2}


\subsection{What is a data type?}

A data type defines the kind of data a variable can hold and the operations that can be performed on it. 

It could be an string like "Hallo" or an interger "3" or a float "3.2" 

Data types can help with structuring data, make working with data more efficient. For example with optimize memory usage so there is allocated the correct amount of space. It can also help with minimaze "Error Prevention" so that two different types can't combine.

an example is from the exercise week 13, question 1. "How do you choose to represent the price of electricity? Describe three possible data structures (using Python) that you could use and explain which one you choose."


the way i would have solved it was



\subsection{Representing colours}


There are multiple ways to repercent color in any program language. Specificily in python there is using a string with the names of the colur for example "Green", "Yellow" or "red" another way is to make a RGB (RedGreenBlue) tuple that store the different values of each colur from 0 to 255 is the full intensity so it could be "color_rgb = (255, 0, 0)  # Red color" , a more simple way is repercent colers as intergers like from 0. So it is numerics, like country 1 get the coulr 1, then country 2 get coulr 2 and maybe country 3 get coulr 1 like the first country. 


in my solution to the colour problem in pyhon, the canExtendColour function would is using interger, if it insted used Strings this part "colouring.get(other) != colour" would still be valid  so there would be no changes, likewise with representing it with RGB tuples, the logic would remain the same. 

explain more here:......



So there are different tradeoffs between choosing this representation. The first on as strings is easy readable for humans, wítch is an advantage, the downside is though that "Predefined list required: You must maintain an array like ["red", "blue", "green", "yellow"].". Then there is the second representation, as Integers there the advantage is (Simple and efficient: Integer comparisons (!=) are very fast.)  .... the disadvantage is that it is Not human-readable and if wanting to intergrate it with libaries like mathplotlib you first have to map the numbers to colure

the last way is RGB, the advantage is that many orther libaryerise is easy to impliment it, and it is highly customizable with the shade for example, and it is dynamic color generation . The disadvantages is though that it is Less human-readable, 

I choose to use intergers in my solution since we dont need to many colurs, 

\subsection{Representing Neighbour Relation}






\subsection{Colouring problem}

To see the full decribtion of my code, see the appendix for the code. 

The first function is "Is_neighbour" that is a function that check if a pair of two counties is neighbours, it does this by a list of tuples of country pairs that is naibours.

the next function is "can_extend_colour" is 



\section{ Question 3}


\subsection{}

See appendix for the full code and the specifications as doc string




In the function "Search" it takes in the starting position as a x,y starting point and an direction from 1 out of the 8 direction eg. (-1,0). Then it seperate it from these inputs, after that it make a list of position from the starting point and in the direction that is given and to the lengh of the word that is giving. So it create a list of the exact location of each letter in the word like [(2, 3), (3, 4), (4, 5), (5, 6)], for the word word = "DIKU".

Then in the next part it uses "return all()" function that only returns True if all parts is true. The first statement is using the privously function to make sure that the word is within the boundry, the next statement is checking ....

nx1 and ny1 is created from the "char in zip(positions,word)) the "zip(position,word)" is a function that macthes the position and each letter in pair. It the example abouve it would be "[((2, 3), 'D'), ((3, 4), 'I'), ((4, 5), 'K'), ((5, 6), 'U')]"


the variable "start" is returning every possible place where the word could start, by returning the position of the first letter "grid[x][y] == word[0]", in our case it would be everywhere there is a "D".




\subsection{}


For this solution i used imperative programming witch is simple and easy for small programs that is not going to be re..... NBNBN blablabla




\subsection{}




To test the code i will use specification driven testing in order to try to break the code. As we learned in the lectures from Friz, that "testing is NOT illustrating that your program runs on a friendly case". I got help from chatGPT to design the test cases.





\section{Question 4}
\subsection{}


Here I define two functions, one called \texttt{calculate\_points} and another called \texttt{total\_points\_from\_file}.

The first function, \texttt{calculate\_points}, takes as input a string.

The second function, \texttt{total\_points\_from\_file}, takes as input the name of a \texttt{.txt} file. First, it initializes a variable \texttt{total} to zero. Then, it uses the built-in \texttt{open()} function on the filename provided as input to the function. It splits each line into a maximum of two parts using the colon (\texttt{:}) as the delimiter, and takes the second element of the list, removes any leading and trailing whitespace using \texttt{.strip()}, and saves this string as \texttt{card\_data}. After this, it appends the value from \texttt{calculate\_points(card\_data)} to \texttt{total}, adding the points from one string. The loop continues until there are no more lines to process, thanks to the \texttt{for line} statement.

\subsection{}

Here I try to use mostly funtional and imperative programming.  

\subsection{}


Explain how you test your program.


\section{}




\subsection{}


\section{Appendix}

\subsection{Question 3}


\lstinputlisting[language=Octave]{scr/diku.py}

\subsection{Question 4}


\lstinputlisting[language=Octave]{scr/cards.py}




\end{document}