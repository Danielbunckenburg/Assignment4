\documentclass{article}
\usepackage{listings}
\usepackage{xcolor}


\definecolor{codegreen}{rgb}{0,0.6,0}
\definecolor{codegray}{rgb}{0.5,0.5,0.5}
\definecolor{codepurple}{rgb}{0.58,0,0.82}
\definecolor{backcolour}{rgb}{0.95,0.95,0.92}

\lstdefinestyle{mystyle}{
    backgroundcolor=\color{backcolour},   
    commentstyle=\color{codegreen},
    keywordstyle=\color{magenta},
    numberstyle=\tiny\color{codegray},
    stringstyle=\color{codepurple},
    basicstyle=\ttfamily\footnotesize,
    breakatwhitespace=false,         
    breaklines=true,                 
    captionpos=b,                    
    keepspaces=true,                 
    numbers=left,                    
    numbersep=5pt,                  
    showspaces=false,                
    showstringspaces=false,
    showtabs=false,                  
    tabsize=2
}

\lstset{style=mystyle}

\begin{document}

\section{Question 1}



https://www.w3schools.com/python/ref_string_find.asp

\section{Question 2}


\subsection{}
A data type is 

Illustrate the role data types play when representing data in Python, using examples from the thursday worksheets in 


\subsection{}

You can repersent couler as number from 1 to n, numbers each number repersent a couler. The problem with this stragerty is that you might run out of destitied colur so it can be hard to tell the couler apart.  


you can also do it by only giving 4 different coulre. 



\newpage

\section{ Question 3}

\subsection{}


First i will use the build in function open(file,mode) where i use the relative file path to the file, so it will work on orther computer. I will call the object file, after that i will call use the build in method read() and call the string content.

Now i will define a function called "CountWord" that takes as input the text, and then the word that would like to be counted.

then i make 2 variables, one called start another called index. 

then i make a while loop, that keeps looping until the condition is false.

the function "text.find()" returns "-1" when there is no match. there by the condition "!= -1" becomes false and the loop stop. 

and then the function returns the count.


\lstinputlisting[language=Octave]{scr/diku.py}

\subsection{}

I used funtional programming as the main paradigm, but there is also a bit of object ortieede programming in the buildin methoes. 


\subsection{}

In my program, I could use the input pattionon




\section{Question 4}
\subsection{}

\lstinputlisting[language=Octave]{scr/cards.py}




\subsection{}

What programming paradigm dominates in your program. Why?

here i use object ortineted programming, also since the build in methoes is used in this way.




\subsection{}

Explain how you test your program.


\section{}

\subsection{}

\end{document}